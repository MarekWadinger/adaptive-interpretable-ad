Anomaly detection systems, able to discriminate abnormal, unexpected patterns and adapt to novel expected patterns in data, are known to be an essential part of risk-averse systems. In particular, anomaly detection systems assess the normal operational conditions allowing Internet of Things (IoT) devices to stream high-fidelity data into control units.

In their highly influential paper, Chandola et al. review former research efforts spanning diverse application domains \cite{Chandola2009}.
Recent studies highlight the need to develop holistic methods with general application and accessible tunability for operators \cite{Laptev2015}, \cite{Kejariwal2015}, \cite{Cook2020}. 

Cook et al. denote substantial aspects that pose challenges to anomaly detection on IoT, namely the context information of the measurement being temporal, spatial, and external, multivariate character, noise, and nonstationarity \cite{Cook2020}. Feature engineering methods allow encoding contextual properties and increase the performance \cite{Fan2019}. However, extensive feature engineering may significantly increase dimensionality, requiring sizeable data storage and high computational resources \cite{Talagala2021}. 

Moreover, nonstationarity resulting from concept drift, an alternation in the pattern of data due to a change in statistical distribution, and change points, permanent changes to the system's state, represents a difficulty of a significant extent. In real-world scenarios, those changes are frequently unpredictable. Therefore, the ability of an anomaly detection method to adapt to changes in the data structure is crucial for long-term deployments. The former scalability problem now introduces a significant latency in detector adaptation \cite{Wu2021}. Incremental learning methods allowed adaptation while restraining the storage of the whole dataset. The supervised operator-in-the-loop solution offered by Pannu et al. showed the detector's adaptation to data labeled on the flight. 
Others approached the problem as sequential processing of bounded data buffers in univariate signals \cite{Ahmad2017134} and multivariate systems \cite{Bosman201514}. 

Lastly, recent efforts to extend anomaly detection tasks to root cause isolation governed the development of explanatory methods capable of diagnosing and tracking faults across the system. Studies can be split into two groups. The first group approaches explainability as the importance of individual features \cite{Carletti2019}, \cite{Nguyen2019}, \cite{Amarasinghe2018}. Those studies allow an explanation of novelty by considering features independently. The second group uses statistical learning creating models explainable via probability. Yang et al. recently proposed a Bayesian network (BN) for fault detection and diagnosis task. Individual nodes of the network represent normally distributed variables, whereas the multiple regression model defines weights and relationships. Using the predefined structure of the BN, the authors propose an offline-trained model with online detection and diagnosis \cite{Yang2022}. Offline training, however, as we wrote earlier, do not allow adaptation to expected novel pattern and, therefore, to our knowledge, is not suitable for long-term operation on real IoT devices.

This paper emphasizes the importance of such adaptability in anomaly detection and proposes a method that addresses this challenge. Here we report the discovery and characterization of an adaptive anomaly detection method for streaming IoT data. The ability to diagnose multivariate data while providing root cause isolation, inherent in the univariate case, extends our previous contribution to the field as presented in \cite{Wadinger2023}. The proposed algorithm represents a general method for a broad range of safety-critical systems where anomaly diagnosis and identification is crucial.

Two case studies show that the proposed method based on dynamic joint normal distribution gives the capacity to explain novelties and isolate the root cause of anomalies and allow adaptation to change points advancing recently developed anomaly detection techniques to the long-term deployment of the service and cross-domain usage. We observe similar detection performance for the cost of lower scalability.

The main contribution of the proposed solution to the developed body of research is that it:
\begin{itemize}
\item Provides both adaptability and interpretability
\item Identifies systematic outliers and root cause
\item Uses self-learning approach on streamed data
\item Utilizes existing IT infrastructure
\item Establishes dynamic limits for signals
\end{itemize}