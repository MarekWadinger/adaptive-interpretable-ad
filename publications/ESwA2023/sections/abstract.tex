In this paper, we introduce an Adaptable and Interpretable Framework for Anomaly Detection (AID) designed for industrial systems utilizing IoT data streams on top of well-established SCADA systems. AID leverages dynamic conditional probability distribution modeling to capture the normal operation of dynamic systems and isolate the root causes of anomalies. The self-supervised framework dynamically updates parameters of underlying model, allowing it to adapt to non-stationarity. AID interprets anomalies as significant deviations from conditional probability, encompassing interactions as well as both spatial and temporal irregularities by exposing them as features. Crucially, AID provides dynamic process limits to integrate with existing alarm handling mechanisms in SCADA and pinpoint root causes at the level of individual inputs. Two industrial-scale case studies showcase AID's capabilities. The first study showcases AID's effectiveness on energy storage system, adapting to changes, setting less conservative process limits for SCADA, and ability to leverage a physical model. The second study monitors battery module temperatures, where AID identifies hardware faults, emphasizing its relevance to energy storage safety and profitability. A benchmark evaluation on real data shows that AID delivers comparable performance to other self-learning adaptable anomaly detection methods, with the significant advancement in diagnostic capabilities for improved system reliability and performance.