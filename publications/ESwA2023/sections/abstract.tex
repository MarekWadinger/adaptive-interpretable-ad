In this paper, we introduce an Adaptable and Interpretable Framework for Anomaly Detection (AID) designed for streaming energy systems utilizing IoT devices. AID leverages dynamic conditional probability distribution modeling to capture the normal operation of dynamic systems and isolate the root causes of anomalies. The self-supervised framework updates parameters of multivariate Gaussian distribution, allowing it to adapt to non-stationarity. AID interprets anomalies as significant deviations from conditional probability, encompassing interactions as well as both spatial and temporal irregularities by exposing them as features. Additionally, dynamic process limits are drawn to pinpoint root causes at the level of individual signals. Two real-world case studies showcase AID's capabilities. The first study showcases AID's effectiveness in Battery Energy Storage Systems, capturing anomalies, setting less conservative process limits, and ability to leverage a physical model. The second study monitors battery module temperatures, where AID identifies hardware faults, emphasizing its relevance to energy storage safety and profitability. A benchmark evaluation on industrial data shows that AID delivers comparable results to other self-learning adaptable anomaly detection methods, with the significant advantage of diagnostic capabilities for improved system reliability and performance. Our framework is openly accessible on \url{https://github.com/MarekWadinger/online_outlier_detection}.% AID represents a significant step forward in adaptable and interpretable anomaly detection for streaming energy systems, offering valuable insights and diagnostic capabilities for improving system reliability and performance.
