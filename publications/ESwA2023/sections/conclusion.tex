In this paper, we demonstrate the capacity of adaptive conditional probability distribution to model the normal operation of dynamic systems employing streaming IoT data and isolate the root cause of anomalies. AID dynamically adapts to non-stationarity by updating multivariate Gaussian distribution parameters over time. Additionally, self-supervision enhances the model by protecting it from the effects of outliers and increasing the speed of adaptation in response to autonomously detected changes in operation.

Our statistical model isolates the root causes of anomalies as extreme deviations from the conditional means vector, considering spatial and temporal effects encoded in features, as demonstrated in our case studies. This approach establishes the system's operational state by analyzing the distribution of signal measurements, computing the distance from the mean of conditional probability, and setting dynamic process limits based on multivariate distribution parameters. Additionally, the detector alerts for non-uniform sampling due to packet drops and sensor malfunctions. These adaptable limits can be seamlessly integrated into SCADA architecture, enhancing context awareness and enabling plug-and-play compatibility with existing infrastructure.

The ability to detect and identify anomalies in the system, isolate the root cause of anomaly to specific signal or feature, and identify signal losses is shown in two case studies on data from operated industrial-scale energy storages. These case studies highlight the model's ability to adapt, diagnose the root cause of anomalies, and leverage both physical models and spatially distributed sensors. Unlike many anomaly detection approaches, the proposed AID method does not require historical data or ground truth information about anomalies, alleviating the general limitations of detection methods employed in the energy industry.

The benchmark performed on industrial data indicates that our model provides comparable results to other self-learning adaptable anomaly detection methods. This is an important property of our model, as it also allows for root cause isolation.

AID represents a significant advancement in the safety and profitability of evolving systems that utilize well-established SCADA architecture and streaming IoT data. By providing dynamic operating limits, AID seamlessly integrates with existing alarm mechanisms commonly employed in SCADA systems. To the best of our knowledge, this contribution is the first of its kind in the field of self-supervised adaptable and interpretable anomaly detection.

Future work on this method will include improvements to the change point detection mechanism, reduction in latency for high-dimensional data, and minimizing the false positive rate, which is a challenge for general plug-and-play models. We will also explore the ability to operate with non-parametric models, in contrast to Gaussian distribution.

Our framework is openly accessible on GitHub at the following URL: \url{https://github.com/MarekWadinger/online_outlier_detection}.

