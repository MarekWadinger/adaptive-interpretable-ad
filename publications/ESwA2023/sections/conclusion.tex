In this paper, we examine the capacity of adaptive conditional probability distribution to model the normal operation of dynamic systems employing streaming IoT devices and isolate the root cause. The dynamics of the systems are elaborated in the model using Welford's online algorithm with the capacity to update and revert sufficient parameters of underlying multivariate Gaussian distribution in time making it possible to elaborate non-stationarity in the process variables. Moreover, the self-supervision allows protection of the distribution from the effect of outliers and increased speed of adaptation in cases of changes in operation.

We assume the Gaussian distribution of measurements over a bounded time frame related to the system dynamics. We consider such an assumption reasonable, with support of multiple trials where the Kolmogorov-Smirnov test did not reject this hypothesis. The statistical model provides the capacity for the interpretation of the anomalies as extremely deviating observations from the mean vector. Another assumption held in this study is that any anomaly, spatial or temporal, can be transformed in such a way that makes it an outlier given that we expose such effects as features as shown in case studies.

Our approach establishes the system's operation state at the global anomaly level by considering interactions between input measurements and engineered features and computing distance from mean of conditional probability. At the second level, dynamic process limits based on PPF at threshold probability, given multivariate distribution parameters, help isolate the root cause of anomalies. This level serves the diagnostic purpose of the model operation. The individual signals contribute to the global anomaly prediction, while the proposed dynamic limits offer less conservative restrictions on individual process operation. In parallel, the detector allows discrimination of signal losses due to packet drops and sensor malfunctioning.

The ability to detect and identify anomalies in the system, isolate the root cause of anomaly to specific signal or feature, and identify signal losses is shown in two case studies on real data. Unlike many anomaly detection approaches, the proposed AID method does not require historical data or ground truth information about anomalies, relieving general limitations. Moreover, it combines adaptability and interpretability, which is an area yet to be explored.

The benchmark performed on industrial data showed the ability to provide comparable results to other self-learning adaptable anomaly detection methods. This is an important property for our model which allows, in addition, the root cause isolation.
The first case study, performed on real operation data of BESS, examined the battery energy storage system and demonstrated the ability to capture system anomalies and provide less conservative limits to signals. The physical model aided decisions about the normality of the measured temperature of BESS.

The second case study exposed the ability to detect anomalies in the temperature profiles of battery modules within the battery pack, considering spatial measurements made by multiple sensors distributed around the module and the average temperature of all the modules within the pack. Hardware fault observed on this deployed device was captured by our model, giving another proof of its importance in energy storage systems monitoring, where tight temperature control plays a significant role in the safety and profitability of the system.

Future works on the method will include improvement to the change point detection mechanism, decrease in the latency on high dimensional data, and false positive rate reduction, from which general plug-and-play models suffer.
