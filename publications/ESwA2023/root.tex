%%
%% Copyright 2007-2020 Elsevier Ltd
%%
%% This file is part of the 'Elsarticle Bundle'.
%% ---------------------------------------------
%%
%% It may be distributed under the conditions of the LaTeX Project Public
%% License, either version 1.2 of this license or (at your option) any
%% later version.  The latest version of this license is in
%%    http://www.latex-project.org/lppl.txt
%% and version 1.2 or later is part of all distributions of LaTeX
%% version 1999/12/01 or later.
%%
%% The list of all files belonging to the 'Elsarticle Bundle' is
%% given in the file `manifest.txt'.
%%
%% Template article for Elsevier's document class `elsarticle'
%% with harvard style bibliographic references

\documentclass[preprint,12pt,authoryear]{elsarticle}

%% Use the option review to obtain double line spacing
%% \documentclass[preprint,review,12pt]{elsarticle}

%% Use the options 1p,twocolumn; 3p; 3p,twocolumn; 5p; or 5p,twocolumn
%% for a journal layout:
%% \documentclass[final,1p,times]{elsarticle}
%% \documentclass[final,1p,times,twocolumn]{elsarticle}
%% \documentclass[final,3p,times]{elsarticle}
%% \documentclass[final,3p,times,twocolumn]{elsarticle}
%% \documentclass[final,5p,times]{elsarticle}
%% \documentclass[final,5p,times,twocolumn]{elsarticle}

%% For including figures, graphicx.sty has been loaded in
%% elsarticle.cls. If you prefer to use the old commands
%% please give \usepackage{epsfig}

%% The amssymb package provides various useful mathematical symbols
\usepackage{amssymb}
%% The amsthm package provides extended theorem environments
%% \usepackage{amsthm}
%% The amsmath package provides equation environment
\usepackage{amsmath}
%% The algorithms
\usepackage{algorithm}
\usepackage{algorithmic}
%% The images
\usepackage{graphicx}
%% Verbatim with URL-sensitive line breaks
\usepackage{url}
% upright sub-index
\newcommand{\ui}[2]{#1 _{\text{#2}}}
% upright sub-index with variable
\newcommand{\uis}[3]{#1 _{\text{#2}, #3}}

%% The lineno packages adds line numbers. Start line numbering with
%% \begin{linenumbers}, end it with \end{linenumbers}. Or switch it on
%% for the whole article with \linenumbers.
\usepackage{lineno}

\journal{Expert Systems with Applications}

\begin{document}

\begin{frontmatter}

%% Title, authors and addresses

%% use the tnoteref command within \title for footnotes;
%% use the tnotetext command for theassociated footnote;
%% use the fnref command within \author or \address for footnotes;
%% use the fntext command for theassociated footnote;
%% use the corref command within \author for corresponding author footnotes;
%% use the cortext command for theassociated footnote;
%% use the ead command for the email address,
%% and the form \ead[url] for the home page:
\title{Adaptable and Interpretable Framework for Anomaly Detection in SCADA-based Industrial Systems}

\author[aff1,aff2]{Marek Wadinger\corref{cor1}}
\ead{marek.wadinger@stuba.sk}
\cortext[cor1]{\textit{Phone numbers:} +421 902 810 324 (Marek Wadinger)}
% TODO: Center the tilde vertically
% \ead[url]{uiam.sk/~wadinger}

\author[aff1,aff2]{Michal Kvasnica}
\ead{michal.kvasnica@stuba.sk}

\affiliation[aff1]{organization={Institute of Information Engineering, Automation and Mathematics, Slovak University of Technology in Bratislava},
    addressline={Radlinskeho 9},
    postcode={812 37},
    state={Bratislava},
    country={Slovakia}
}
\affiliation[aff2]{organization={Tesla Labs s.r.o.},
    addressline={Pálenica 53/79},
    postcode={033 17},
    state={Liptovský Hrádok},
    country={Slovakia}
}

\begin{abstract}
%% Text of abstract
This paper presents the Real-time Adaptive and Interpretable Detection (RAID) algorithm. The novel approach addresses the limitations of state-of-the-art anomaly detection methods for multivariate dynamic processes, which are restricted to detecting anomalies within the scope of the model training conditions. The RAID algorithm adapts to non-stationary effects such as data drift and change points that may not be accounted for during model development, resulting in prolonged service life. A dynamic model based on joint probability distribution handles anomalous behavior detection in a system and the root cause isolation based on adaptive process limits. RAID algorithm does not require changes to existing process automation infrastructures, making it highly deployable across different domains. Two case studies involving real dynamic system data demonstrate the benefits of the RAID algorithm, including change point adaptation, root cause isolation, and improved detection accuracy.
\end{abstract}

%%Graphical abstract
\begin{graphicalabstract}
    \includegraphics[width=\columnwidth]{figures/ESwA23 - Graphical Abstract.pdf}
\end{graphicalabstract}

%%Research highlights
\begin{highlights}
\item Interpretable anomaly detector with self-supervised adaptation
\item Demonstrates interpretability by providing dynamic operating limits
\item Leverages self-learning approach on streamed IoT data
\item Utilizes existing SCADA-based industrial infrastruture
\item Offers faster response time to incidents due to root cause isolation
\end{highlights}

\begin{keyword}
%% keywords here, in the form: keyword \sep keyword
Anomaly detection \sep Root cause isolation \sep Iterative learning \sep Statistical learning \sep Self-supervised learning
\end{keyword}

\end{frontmatter}

\linenumbers

%% main text
\section{Introduction}
\label{Introduction}
Anomaly detection systems play a critical role in risk-averse systems by identifying abnormal patterns and adapting to novel expected patterns in data. These systems are particularly vital in the context of Internet of Things (IoT) devices that continuously stream high-fidelity data to control units.

In this rapidly evolving field, Chandola et al. conducted an influential review of prior research efforts across diverse application domains \cite{Chandola2009}.
Recent studies have underscored the need for holistic and tunable anomaly detection methods accessible to operators(\cite{Laptev2015, Kejariwal2015, Cook2020}).

Cook et al. denote substantial aspects that pose challenges to anomaly detection on IoT, including the temporal, spatial, and external context of measurements, multivariate characteristics, noise, and nonstationarity (\cite{Cook2020}). Feature engineering methods allow the encoding of contextual properties and enhance the performance (\cite{Fan2019}). However, extensive feature engineering may significantly increase dimensionality, requiring sizeable data storage and high computational resources (\cite{Talagala2021}).

Moreover, nonstationarity resulting from concept drift, an alternation in the pattern of data due to a change in statistical distribution, and change points, permanent changes to the system's state, represents a difficulty of a significant extent (\cite{Salehi2018}). In real-world scenarios, those changes are frequently unpredictable in their spatial and temporal characteristics and require systems with solid outlier rejection properties of intelligent tracking algorithms (\cite{Barbosa2019162}). Therefore, the ability of an anomaly detection method to adapt to changes in the data structure is crucial for long-term deployments. Nevertheless, as (\cite{Tartakovsky2013}) remarked, instantaneous detection is not an option, unless the false alarm risk is high

The former scalability problem now introduces a significant latency in detector adaptation (\cite{Wu2021}). Incremental learning methods allowed adaptation while restraining the storage of the whole dataset. The supervised operator-in-the-loop solution offered by Pannu et al. showed the detector's adaptation to data labeled on the flight (\cite{Pannu2012}).
Others approached the problem as sequential processing of bounded data buffers in univariate signals (\cite{Ahmad2017134}) and multivariate systems (\cite{Bosman201514}).

\subsection{Related Work}
Recent research has extended the scope of anomaly detection tasks to include root cause isolation governed by the development of explanatory methods capable of diagnosing and tracking faults across the system. Studies can be split into two groups of distinct approaches. The first group approaches explainability as the importance of individual features (\cite{Carletti2019}), (\cite{Nguyen2019}), (\cite{Amarasinghe2018}). Those studies allow an explanation of novelty by considering features independently. The second group uses statistical learning creating models explainable via probability. Yang et al. recently proposed a Bayesian network (BN) for fault detection and diagnosis tasks. Individual nodes of the network represent normally distributed variables, whereas the multiple regression model defines weights and relationships. Using the predefined structure of the BN, the authors propose an offline-trained model with online detection and diagnosis (\cite{Yang2022}). Offline training, however, as we wrote earlier, do not allow adaptation to expected novel pattern and, therefore, to our knowledge, is not suitable for long-term operation on real IoT devices.

This paper emphasizes the importance of combining adaptability in interpretable anomaly detection and proposes a method that addresses this challenge. Here we report the discovery and characterization of an adaptive anomaly detection method for streaming IoT data. The ability to diagnose multivariate data while providing root cause isolation, inherent in the univariate case, extends our previous contribution to the field as presented in (\cite{Wadinger2023}). The proposed algorithm represents a general method for a broad range of safety-critical systems where anomaly diagnosis and identification are paramount.

\subsection{Novelty of proposed approach}
The idea of using statistical outlier detection is well-established. We highlighting impactful contributions of (\cite{Yamanishi2002}) and (\cite{Yamanishi2004}). The authors propose a method for detecting anomalies in a time series. The method is based on the assumption that the continuous data is generated by a mixture of Gaussian distributions, while discrete data is modeled as histogram density. The authors solve the problem of change point detection as well. However, the adaptation system is unaware of such changes, making the moving window the only source of adaptation. Our self-supervised approach offers intelligent adaptation w.r.t. detected change points. Moreover, the author of the study does not attempt to isolate the root cause of the anomaly. We do so by computing the conditional probability of each measurement given the rest of the measurements and drawing limits defining the normal event probability threshold.

A limited number of studies have focused on adaptation and interpretability within the framework of anomaly detection. Two recent contributions in this area are (\cite{Steenwinckel2018}) and (\cite{Steenwinckel2021}). In (\cite{Steenwinckel2018}), the authors emphasize the importance of combining prior knowledge with a data-driven approach to achieve interpretability, particularly concerning root cause isolation. They propose a novel approach that involves extracting features based on knowledge graph pattern extraction and integrating them into the anomaly detection mechanism. This graph is subsequently transformed into a matrix, and adaptive region-of-interest extraction is performed using reinforcement learning techniques. To enhance interpretability, a Generative Adversarial Network (GAN) reconstructs a new graphical representation based on selected vectors. However, it's important to note that the validation of this idealized approach is pending further investigation. Lately, (\cite{Steenwinckel2021}) introduced a comprehensive framework for adaptive anomaly detection and root cause analysis in data streams. While the adaptation process is driven by user feedback, the specific mechanism remains undisclosed. The authors present an interpretation of their method through a user dashboard, featuring visualizations of raw data. This dashboard is capable of distinguishing between track-related problems and train-related issues, based on whether multiple trains at the same geographical location approach the anomaly. Meanwhile, our attempts aim to develop a self-supervised method capable of learning without human supervision which is often limited in time and poses significant delays in adaptation, while interpretation offers straightforward statistical reasoning and root cause isolation.

\subsection{Validation}\label{par:validation}
Two case studies show that our proposed method, based on dynamic joint normal distribution, has the capacity to explain novelties, isolate the root cause of anomalies, and allow adaptation to change points, advancing recently developed anomaly detection techniques for long-term deployment and cross-domain usage. We observe similar detection performance, albeit with lower scalability, when comparing our approach to well-established unsupervised anomaly detection methods in streamed data which create a bedrock for many state-of-the-art contributions, such as One-Class SVM (\cite{Amer2013, Liu2014, Krawczyk2015, Miao2019, Gozuacik2021}), and Half-Space Trees (\cite{Wetzig2019, Lyu2020}).

\subsection{Broader Impact}
Potential applications of the proposed method are in the field of energy storage systems, where the ability to detect anomalies and isolate their root cause, whilst adapting to changes in operation and environment, is crucial for the safety of the system. The proposed method is suitable for the existing infrastructure of the system, allowing detection and diagnosis of the system based on existing data streams. The dynamic process limits allow operational metrics monitoring, making potential early detection and prevention easier. Using adaptable methods without interpretability, on the other hand, may pose safety risks and lower total financial benefits, as the triggered false alarms may need to be thoroughly analyzed, resulting in prolonged downtimes.

\subsection{Paper Organization}
The paper is structured as follows: We begin with the problem and motivation in \textbf{Section \ref{Introduction}}, providing context. Next, in \textbf{Section \ref{Preliminaries}}, we lay the theoretical groundwork. Our proposed adaptive anomaly detection method is detailed in \textbf{Section \ref{Proposed Method}}. We then demonstrate real-world applications in \textbf{Section \ref{Case Study}}. Finally, we conclude the paper in \textbf{Section \ref{Conclusion}}, summarizing findings and discussing future research directions.

The main contribution of the proposed solution to the developed body of research is that it:
\begin{itemize}
\item Enriches interpretable anomaly detection with adaptive capabilities
\item Identifies systematic outliers and root cause
\item Uses self-learning approach on streamed data
\item Utilizes existing IT infrastructure
\item Establishes dynamic limits for signals
\end{itemize}


\section{Preliminaries}
\label{Preliminaries}
In this section, we present the fundamental ideas that form the basis of the developed approach. Subsection \ref{AA:Welford} explains Welford's online algorithm, which can adjust distribution to changes in real time. Subsection \ref{AA:InvWelford} proposes a two-pass implementation that can reverse the impact of expired samples. The math behind distribution modeling in Subsection \ref{AA:Distribution} establishes the foundation for the Gaussian anomaly detection model discussed in the final Subsection \ref{AA:Anomaly} of the preliminaries.

\subsection{Welford's Online Algorithm}\label{AA:Welford}
Welford introduced a numerically stable online algorithm for calculating mean and variance in a single pass. The algorithm allows the processing of IoT device measurements without the need to store their values \cite{Wel62}.

Given measurement \(x_i\) where \(i=1,...,n\) is a sample index in sample population \(n\), the corrected sum of squares \(S_n\) is defined as
\begin{equation}
S_n = \sum_{i=1}^n (x_i - \bar x_n)^2\text{,}\label{eq:sumsquares}
\end{equation}
with the running mean \(\bar x_n\) defined as previous mean \(\bar x_{n-1}\) weighted by proportion of previously seen population \(n-1\) corrected by current sample as
\begin{equation}
\bar x_n = \frac{n-1}{n} \bar x_{n-1} + \frac{1}{n}x_n = \bar x_{n-1} + \frac{x_n - \bar x_{n-1}}{n}\text{.}\label{eq:runmean}
\end{equation}
Throughout this paper, we consider a following formulation of an update to the corrected sum of squares:
\begin{equation}
S_n = S_{n-1} + (x_n - \bar x_{n-1})(x_n - \bar x_n)\text{,}\label{eq:upsumsquares}
\end{equation}
as it is less prone to numerical instability due to catastrophic cancellation. Finally, the corresponding unbiased variance is
\begin{equation}
s^2_n = \frac{S_{n}}{n-1}\text{.}\label{eq:runvar}
\end{equation}

This implementation of the Welford method requires the storage of three scalars: \(\bar x_{n-1}\); \(n\); \(S_n\).

\subsection{Inverse Welford's Algorithm}\label{AA:InvWelford}
Based on \eqref{eq:runmean}, it is clear that the influence of the latest sample over the running mean decreases as the population \(n\) grows. For this reason, regulating the number of samples used for sample mean and variance computation has crucial importance over adaptation. Given access to the instances used for computation and expiration period \(t_e \in \mathbb{N}_{0}^{n-1}\), reverting the impact of \(x_{n-t_e}\) can be written as follows

\begin{equation}
S_{n-1} = S_n - (x_{n-t_e} - \bar x_{n-1})(x_{n-t_e} - \bar x_n)\text{,}\label{eq:revrunmean}
\end{equation}

where the reverted mean is given as

\begin{equation}
\bar x_{n-1} = \frac{n}{n-1} \bar x_{n} - \frac{1}{n-1}x_{n-t_e} = \bar x_{n} - \frac{x_{n-t_e} - \bar x_{n}}{n-1}\text{.}\label{eq:revmean}
\end{equation}


Finally, the unbiased variance follows the formula:

\begin{equation}
s^2_{n-1} = \frac{S_{n-1}}{n-2}\text{.}\label{eq:revvar}
\end{equation}

\subsection{Statistical Model of Multivariate System}\label{AA:Distribution}
Multivariate normal distribution generalizes the multivariate systems to the model where the degree to which variables are related is represented by the covariance matrix. Gaussian normal distribution of variables is a reasonable assumption for process measurements, as it is a common distribution that arises from stable physical processes measured with noise. The general notation of multivariate normal distribution is:
\begin{equation}\mathbf{X}\ \sim\ \mathcal{N}_k(\boldsymbol\mu,\, \boldsymbol\Sigma)\text{,}
\end{equation}

where $k$-dimensional mean vector is denoted as \(\boldsymbol\mu = (\bar x_{1},...,\bar x_{k})^T\ \in \mathbb{R}^{k}\) and \(\boldsymbol\Sigma \in \mathbb{R}^{k\times{k}}\) is the $k \times k$ covariance matrix, where \(k\) is the index of last random variable.

The probability density function (PDF) \(f(\boldsymbol{x}; \boldsymbol{\mu}, \boldsymbol{\Sigma})\) of multivariate normal distribution is denoted as:
\begin{equation}
f(\boldsymbol{x}; \boldsymbol{\mu}, \boldsymbol{\Sigma}) = \frac{1}{(2\pi)^{k/2} |\boldsymbol{\Sigma}|^{1/2}} e^{-\frac{1}{2} (\boldsymbol{x}-\boldsymbol{\mu})^\top \boldsymbol{\Sigma}^{-1} (\boldsymbol{x}-\boldsymbol{\mu})}\text{,}
\end{equation}

where $\boldsymbol{x}$ is a $k$-dimensional vector of measurements $x_i$ at time $i$, $|\boldsymbol{\Sigma}|$ denotes the determinant of $\boldsymbol{\Sigma}$, and $\boldsymbol{\Sigma}^{-1}$ is the inverse of $\boldsymbol{\Sigma}$.

The cumulative distribution function (CDF) of a multivariate Gaussian distribution describes the probability that all components of the random matrix \(\boldsymbol{X}\) take on a value less than or equal to a particular point \(\boldsymbol{x}\) in space, and can be used to evaluate the likelihood of observing a particular set of measurements or data points. The CDF is often used in statistical applications to calculate confidence intervals, perform hypothesis tests, and make predictions based on observed data. In other words, it gives the probability of observing a random vector that falls within a certain region of space. The standard notation of CDF is as follows:

\begin{equation}
F(\boldsymbol{x}; \boldsymbol{\mu}, \boldsymbol{\Sigma}) = \int_{-\infty}^{\boldsymbol{x}} f(\boldsymbol{x}; \boldsymbol{\mu}, \boldsymbol{\Sigma})  \text{d}\boldsymbol{x}\text{,}\label{eq:cdf}
\end{equation}

where $\text{d}\boldsymbol{x}$ denotes the integration over all $k$ dimensions of $\boldsymbol{x}$.

As the equation \eqref{eq:cdf} cannot be integrated explicitly, an algorithm for numerical computation was proposed in \cite{Genz2000}.


Given the PDF, we can also determine the value of \(\boldsymbol{x}\) that corresponds to a given quantile $q$  using a numerical method for inversion of CDF (ICDF) often denoted as percent point function (PPF) or $F(\boldsymbol{x}; \boldsymbol{\mu}, \boldsymbol{\Sigma})^{-1}$. An algorithm that calculates the value of the PPF for univariate normal distribution is reported below as Algorithm \ref{alg:ppf}.

\begin{algorithm}[H]
\caption{{Percent-Point Function for Normal Distribution}} \label{alg:ppf}
 \begin{algorithmic}[1]
 \renewcommand{\algorithmicrequire}{\textbf{Input:}}
 \renewcommand{\algorithmicensure}{\textbf{Output:}}
 \REQUIRE quantile $q$, sample mean $\bar x_n$ \eqref{eq:runmean}, sample variance $s^2_n$ \eqref{eq:runvar}
 \ENSURE  threshold value $x_{n,q}$
 \\ \textit{Initialisation} :
  \STATE $f \leftarrow 10$; $l \leftarrow -f $; $r \leftarrow f;$
 \\ \textit{LOOP Process}
  \WHILE {$F(l; \bar x_n, s^2_n) > 0$}
  \STATE $r \leftarrow l $;
  \STATE $l \leftarrow lf $;
  \ENDWHILE
  \WHILE {$F_X(r)-q < 0$}
    \STATE $l \leftarrow r $;
    \STATE $r \leftarrow rf $;
  \ENDWHILE
  \STATE {$\tilde{x}_{n,q} = \text{arg} \min_{x_n} \| F(x_n; \bar x_n, s^2_n) - q \| ~ \text{s.t.} ~ l \le x_n \le r$}
 \RETURN $\tilde{x}_{n,q}  \sqrt{s^2_n} + \bar x_n $
 \end{algorithmic}
\end{algorithm}

The Algorithm \ref{alg:ppf} for PPF computation is solved using an iterative root-finding algorithm such as Brent's method \cite{Brent72}.

\subsection{Multivariate Gaussian Anomaly Detection}\label{AA:Anomaly}
From a statistical viewpoint, outliers can be denoted
as values that significantly deviate from the mean. Assuming that the spatial and temporal characteristics of the system over the moving window can be encoded as normally distributed features, we can claim, that any anomaly may be detected as an outlier.

In empirical fields, such as machine learning, the three-sigma rule ($3\sigma$) defines a region of distribution where normal values are expected to occur with near certainty. This assumption makes approximately 0.27\% of values in the given distribution considered anomalous.

The \(3\sigma\) rule establishes the probability that any sample \(\boldsymbol{x_i}\) of a random variable \(\boldsymbol{X}\) lies within a given CDF over a semi-closed interval as the distance from the sample mean \(\boldsymbol{\mu}\) of 3 sample standard deviations \(\boldsymbol{\Sigma}\) and gives an approximate value of $q$ as
\begin{equation}
q=P\{|\boldsymbol{x_i}-\boldsymbol{\mu}|<3\boldsymbol{\Sigma}\}=0.99730\text{.}
\end{equation}

Using a probabilistic model of normal behavior lets us query the threshold vectors \(\boldsymbol{x_{l}}\) and \(\boldsymbol{x_{u}}\) which corresponds to the closed interval of CDF at which probability was established. Inversion of \eqref{eq:cdf} can be used for such query resulting in:

\begin{equation}
\boldsymbol{x_l} = F((1 - P\{|\boldsymbol{x_i}-\boldsymbol{\mu}|<3\boldsymbol{\Sigma}\circ\boldsymbol{I}\}); \boldsymbol{\mu}, \boldsymbol{\Sigma}\circ\boldsymbol{I})^{-1}\text{,}\label{eq:thresh_low}
\end{equation}

for the lower limit, and

\begin{equation}
\boldsymbol{x_u} = F((P\{|\boldsymbol{x_i}-\boldsymbol{\mu}|<3\boldsymbol{\Sigma}\circ\boldsymbol{I}\}); \boldsymbol{\mu}, \boldsymbol{\Sigma}\circ\boldsymbol{I})^{-1}\text{,}\label{eq:thresh_high}
\end{equation}

for upper one, where $\boldsymbol{\Sigma}\circ\boldsymbol{I}$ represents diagonal elements of $\boldsymbol{\Sigma}$.

However, the problem of computing CDF of a multivariate normal distribution is that it may result in numerical issues for small probabilities. To avoid underflow, the logarithm of CDF (log-CDF) is computed, converting the product of individual elements into a numerically more stable summation. The value of $T$ represents a threshold, defining the discrimination boundary between normal operation and anomaly. The predicted state of the system $Y_i$ at time $i$ is defined as
\begin{equation}
y_i =
  \begin{cases}
     0 & \text{ if } T \leq \log{F(\boldsymbol{x_i}; \boldsymbol{\mu}, \boldsymbol{\Sigma})} \\
     1 & \text{ if }  T > \log{F(\boldsymbol{x_i}; \boldsymbol{\mu}, \boldsymbol{\Sigma})}\text{,}\label{eq:anomaly}
  \end{cases}
\end{equation}

where $y_i = 0$ for normal operation of the system and $y_i = 1$ for anomalous operation.


\clearpage
\section{Adaptive Anomaly Detection and Interpretation Framework}
\label{Proposed Method}
We suggest a novel approach to provide dynamic process limits using an online outlier detection algorithm capable of handling concept drifts in real-time. Our main contribution is based on using an inverse cumulative distribution function (ICDF) to supply a dynamic real-valued threshold for anomaly detection, i.e., to find the values of the signal which corresponds to the alert-triggering process limits. Therefore, in the context of machine learning, we are tackling an inverse problem, i.e., calculating the input that produced the observation. To utilize an adaptive ICDF-based threshold system, the univariate Gaussian distribution has to be fitted to the data in online training and ICDF evaluated on the fly. It is important to note that the analysis is based on the assumption that the data collected over moving windows follow a Gaussian normal distribution, rather than assuming that the data over the entire observed period follows this distribution. Thus, the influence of trends in the data can be mitigated by selecting the appropriate window size. 
This method is divided into four parts and described in the following lines. For a simplified representation of the method see Algorithm \ref{alg:detector}.

\subsection{Model Initialization}\label{init}
The initial conditions of the model parameters are \(\mu_0 = x_0\) for mean and \(s^2_0 = 1\) for variance. The score threshold $q$ is constant and set to $3\sigma$. Moreover, there are two user-defined parameters: the expiration period $t_e$, and the time constant of the system $t_c$. The expiration period, which defines the period over which the time-rolling computations are performed, can be altered to change the proportion of expected anomalies and allows relaxation (longer expiration period) or tightening (shorter expiration period) of the thresholds. The time constant of the system determines the speed of change point adaptation as it influences the selection of anomalous points that will be used to update the model for a window of values \(Y=\{y_{i-t_c},...,y_{i}\}\) if the following condition holds true
\begin{equation}
{\frac{\sum_{y\in Y}y}{n(Y)}} > q\text{,}\label{eq:condition}
\end{equation}
where \(n(Y)\) represents dimensionality of \(Y\).

The existence of two tunable and easy-to-interpret hyper-parameters makes it very easy to adapt the solution to any univariate anomaly detection problem.

\subsection{Online training}\label{train}
Training of the model takes place in an online fashion, i.e., the model learns one sample at a time at the moment of its arrival. Learning updates the mean and variance of the underlying Gaussian distribution. The computation of moving mean \eqref{eq:runmean} and variance \eqref{eq:runvar} is handled by Welford's method. Each sample after the expiration period is forgotten and its effect reverted in the second pass. First, the new mean is computed using \eqref{eq:revmean} which accesses the first value in the bounded buffer. The value is dropped in the same pass. Second, the new sample variance is reverted based on \eqref{eq:revvar} using the new mean and current mean that is overwritten afterward. For details see Subsection \ref{AA:InvWelford}.

\subsection{Online prediction}\label{predict}
In the prediction phase, \(z\)-score \eqref{eq:zscore} is computed and passed through $E_A$ \eqref{eq:erf} in order to evaluate $F_{X}(x_i)$ from \eqref{eq:cdf}. The algorithm marks the incoming data points if their corresponding anomaly score from \eqref{eq:score} is out of the range defined by threshold \(q\). In other words, it marks signal value \(x_i\) that is higher or equal to the threshold, which bounds the $3\sigma$ region.
%, i.e., the probability, that the next value \(x\) from measurement real-valued random variable \(X\) will be inside the bounded region \(x\).

\subsection{Dynamic Process Limits}\label{constrait}
Normal process operation is constrained online using ICDF. The constant value of \(q\) and parameters of the fitted distribution are both passed through Algorithm \ref{alg:ppf} to obtain value, which corresponds to the value of \(x\) that would trigger an upper bound outlier alarm at the given time instance. To obtain a lower bound of operation conditions the same procedure is applied to the distribution fitted on negative values of input.

\begin{algorithm}[H]
\caption{{Online Anomaly Detection Workflow}} \label{alg:detector}
 \begin{algorithmic}[1]
  \renewcommand{\algorithmicrequire}{\textbf{Input:}}
  \renewcommand{\algorithmicensure}{\textbf{Output:}}
  \REQUIRE expiration period $t_e$, time constant $t_c$
  % sample mean $\bar x_0$, sample variance $s^2_0$, 
  \ENSURE  score $y_i$, threshold $x_{i,q}$
 \\ \textit{Initialisation} : 
  \STATE $i \leftarrow 1;~ n \leftarrow 1;~ q \leftarrow 0.9973;~ \bar x  \leftarrow x_0;~  s^2 \leftarrow 1$;
  \STATE compute $F_X(x_0)$ using \eqref{eq:zscore};
 \\ \textit{LOOP Process}
  \LOOP
    \STATE {$x_i \leftarrow$ RECEIVE()};
    \STATE $y_i \leftarrow$ PREDICT($x_i$) using \eqref{eq:score};
    \STATE $x_{i,q} \leftarrow$ GET($q, \bar x, s^2$) using Algorithm \ref{alg:ppf};
    \IF {\eqref{eq:score_norm} \OR \eqref{eq:condition}}
     \STATE {$\bar x$, $s^2 \leftarrow$ UPDATE($x_i, \bar x, s^2, n$) using \eqref{eq:runmean}, \eqref{eq:runvar}};
     \STATE $n \leftarrow n + 1$;
     \FOR {$x_{i-t_e}$}
      \STATE {$\bar x$, $s^2 \leftarrow$ REVERT($x_{i-t_e}, \bar x, s^2, n$) using \eqref{eq:revmean}, \eqref{eq:revvar}};
      \STATE $n \leftarrow n - 1$;
     \ENDFOR
    \ENDIF
    \STATE $i \leftarrow i + 1$;
  \ENDLOOP
 \end{algorithmic}
\end{algorithm}

\clearpage
\section{Case Study}
\label{Case Study}
This section provides a benchmark and two case studies that showcase the effectiveness and applicability of our proposed approach. In the following Subsections, we investigate the properties and performance of the approach using streamed benchmark system data and signals from IoT devices in a microgrid system. The successful deployment demonstrates that this approach is suitable for existing process automation infrastructure.

The case studies were realized using Python 3.10.1 on a machine employing an 8-core Apple M1 CPU and 8 GB RAM.

\subsection{Benchmark}\label{AA:Benchmark}
In this subsection, we compare the proposed method with adaptive unsupervised detection methods without an interpretability layer. Two of the well-established methods, providing iterative learning capabilities over multivariate time-series data are One-Class Support Vector Machine (OC-SVM) and Half Spaced Trees (HS-Trees). Both methods represent the backbone of multiple state-of-the-art methods for cases of anomaly detection on dynamic system data, as we brief listed in Introduction \ref{par:validation}.

Comparison is conducted on real benchmarking data, annotated with labels of whether the observation was anomalous or normal. The dataset of Skoltech Anomaly Benchmark (SKAB) \cite{skab2020} is used for this purpose, as no established benchmarking multivariate data were found regarding energy storage systems. It represents a combination of experiments with the behavior of rotor imbalance as a subject to various functions introduced to control action as well as slow and sudden changes in the amount of water in the circuit. The system is described by 8 features and conveys slow and sudden drifts.

The data were preprocessed according to best practices for the given method, namely: standard scaling for OC-SVM, normalization for HS-Trees, while no scaling was required by our proposed method. Preprocessing is performed online as it would be in the production environment, with running mean and variance used in online standard scaler, while normalization employs running peak-to-peak distance. As stated in the employed library for the online machine learning river, such processing has no detrimental effect on performance in the long run (\cite{Montiel2021})

The optimal hyperparameters for both reference methods is found using Bayesian Optimization. Due to no further knowledge about the data generating process, and equity in benchmark, the hyperparameters of our proposed method were optimized using Bayesian Optimization as well. 20 steps of random exploration with 100 iterations of Bayesian Optimization were used, increasing default values set in the Bayesian Optimization library (\cite{Nogueira2014}).

The hyperparameters are optimized with F1 score as cost function first, to maximize both precision and recall on anomalous samples. Second, the hyperparameters are optimized with macro F1 score, as it considers performance on both anomalous samples as well as normal samples equally. Therefore, the performance is not indifferent towars type I. errors, false alarms due to wrong detection of normal data as anomalies.

As adaptation is required and anticipated within benchmark datasets, the performance is evaluated iteratively, similarly to the operation after deployment. The metric is updated with each new sample and its final value used to drive Bayesian Optimization. The performance is evaluated using the best performing model, found by Bayesian Optimization. The performance of the proposed method is evaluated on the same data.

Hyperparameter search ranges were specified regardless of the data domain. The ranges in both cases of OC-SVM and HS-Trees were centered around the default values of the library. The ranges for the proposed method were set arbitrarily wide, to allow the Bayesian Optimization to explore the space of hyperparameters. Values of quantile filter threshold were aligned with the threshold used in our proposed method. The ranges are provided in Table \ref{tab:hyperparam_ranges}.

\begin{table}[ht]
\centering
\caption{Hyperparameter Ranges for Detection Algorithms}
\label{tab:hyperparam_ranges}
\begin{tabular}{|l|l|c|c|}
\hline
\textbf{Algorithm} & \textbf{Hyperparameters} & \textbf{Default} & \textbf{Ranges} \\
\hline
AIM &
\begin{tabular}[c]{@{}l@{}}Threshold\\$t_e$\\$t_a$\\Grace Period\end{tabular} &
\begin{tabular}[c]{@{}c@{}}0.99735\\-\\$t_e$\\$t_e$\end{tabular} &
\begin{tabular}[c]{@{}c@{}}(0.85, 0.99993)\\(150, 500)\\(50, 1000)\\(50, 1000)\end{tabular} \\
\hline
OC-SVM &
\begin{tabular}[c]{@{}l@{}}Threshold\\Learning Rate\end{tabular} &
\begin{tabular}[c]{@{}c@{}}-\\0.01\end{tabular} &
\begin{tabular}[c]{@{}c@{}}(0.85, 0.99993)\\(0.005, 0.02)\end{tabular} \\
\hline
HS-Trees &
\begin{tabular}[c]{@{}l@{}}Threshold\\N Trees\\Max Height\\Window Size\end{tabular} &
\begin{tabular}[c]{@{}c@{}}-\\10\\8\\250\end{tabular} &
\begin{tabular}[c]{@{}c@{}}(0.85, 0.99993)\\(5, 15)\\(6, 10)\\(200, 300)\end{tabular} \\
\hline
\end{tabular}
\end{table}


The results are provided in Table \ref{tab:perf_comp}, evaluating F1 score, Recall and Precision. A value of 100\% at each metric represents a perfect detection. The latency represents the average computation time per sample of the pipeline including training and data preprocessing.

\begin{table}[htbp]
\begin{center}
\begin{tabular}{|l|c|c|c|c|}
    \hline
    \textbf{Algorithm} & F1 [$\%$] & Recall [$\%$] & Precision [$\%$] & Avg. Latency [ms] \\
    \hline
    AID & $\boldsymbol{48.70}$ & 49.90 & $\boldsymbol{47.56}$ & 1.55 \\
    \hline
    OC-SVM & 44.42 & $\boldsymbol{56.67}$ & 36.52 & 0.44 \\
    \hline
    HS-Trees & 34.10 & 32.57 & 35.77 & $\boldsymbol{0.21}$ \\
    \hline
\end{tabular}
\end{center}
\end{table}

The results in Table \ref{tab:perf_comp} suggest, that our algorithm provides slightly better performance than reference methods. Based on the Scoreboard for various algorithms on SKAB's Kaggle page, our iterative approach performs comparably to the evaluated batch-trained model. Such a model has all the training data available before prediction unlike ours, evaluating the metrics iteratively on a streamed dataset.

\subsection{Battery Energy Storage System (BESS)}\label{AA:BESS}
In the first case study, we verify our proposed method on BESS. The BESS reports measurements of State of Charge (SoC), supply/draw energy set-points, and inner temperature, at the top, middle, and bottom of the battery module. Tight battery cell temperature control is needed to optimize performance and maximize the battery's lifespan. Identifying anomalous events and removal of corrupted data might yield significant improvement in the process control level and increase the reliability and stability of the system.

The default sampling rate of the signal measurement is 1 minute. However, network communication of the IoT devices is prone to packet dropout, which results in unexpected non-uniformities in sampling. The data are normalized to the range $[0, 1$] to protect the sensitive business value. The proposed approach is deployed to the existing infrastructure of the system, allowing real-time detection and diagnosis of the system.

The industrial partner provided a physical model of the battery cell temperature, defined as follows:
\begin{align}
 \uis{T}{bat}{i+1} &= \uis{T}{bat}{i} + \ui{T}{s} (\ui{q}{fan} \ui{V}{b,max} \rho \ui{c}{p} (\ui{T}{out} - \uis{T}{bat}{i}) + \ui{V}{c,max}\ui{q}{circ.fan} \rho \ui{c}{p} \uis{T}{bat}{i} \nonumber \\
 &+ \ui{q}{circ.fan} (\ui{P}{cool} \ui{q}{cool} \ui{P}{heat} \ui{q}{heat}) + \ui{c}{scale} \ui{Q}{bat} + \ui{q}{inner fans} \\
 &- (\ui{V}{b,max} \ui{q}{fan} \ui{V}{c,max} \ui{q}{circ.fan}) \rho \ui{c}{p} \uis{T}{bat}{i}) / (\ui{m}{bat} \ui{c}{p,b}) \nonumber
\end{align}

When combined with an averaged measurement of battery cell temperature, we could compute the difference between real and predicted temperature. Such deviation can be useful in detecting unexpected patterns in temperature. Nevertheless, it may be inaccurate as the physical model is simplified and does not account for spatial aspects, like temperature gradients as well as different dynamic effects of charging and discharging on temperature. For instance, in Fig. \ref{fig:bess} mainly during the grace period we see, that the dynamics of cooling is not captured well, resulting in subtle positive difference between average cell temperature and the temperature predicted by the model. Therefore, the raw measured temperature is used as well.
The deviation between demanded power and delivered power was used to aid the identification of the state, as the increased difference might be related to other unexpected and novel patterns.

\begin{figure}[htbp]
\centerline{\includegraphics{figures/BESS_thresh.pdf}}
\caption{Time Series of BESS measurements (blue line) of process variables. The y-axis renders the values after the normalization of raw inputs. Root causes of anomalies are marked within specific signals as red dots. The light red area represents out-of-limits values for individual signals. Non-uniform sampling is marked as red bars. Green bars represent the times, at which changepoint was detected. All the signal anomalies are depicted as orange bars below the graph.}
\label{fig:bess}
\end{figure}

Fig. \ref{fig:bess} depicts the operation of the BESS over March 2022. Multiple events of anomalous behavior happened within this period, confirmed by the operators, that are observable through a sudden or significant shift in measurements in a given period. As the first step, the detection mechanism was initialized, following the provided guidelines for parameter selection in Subsection \ref{init}. The expiration period was set to $\ui{t}{e} = 7$ days, due to the weekly seasonality of human behavior impacting battery usage. The threshold was kept at default value $T = 0.99735$. A grace period, during which the model learns from both normal and anomalous data (though normal are expected, yet not required here), is shortened to 2.5 days to observe detectors reaction to the effect of tests performed on BESS happening on 3\textsuperscript{rd} day from deployment of the system.

As changepoint adaptation in presence of anomalies follows \eqref{eq:condition}, the tests on 3\textsuperscript{rd} day triggered adaptation, resulting in increased variance of distribution concerning Average Cell Temperature. The increased variance is observable in the light red area, loosening the region of normal operation.

The deployment and operation of the anomaly detection system demonstrate adaptation to changepoint on 7\textsuperscript{th} March 2022 that appeared due to the relocation of the battery storage system outdoors. The model adapted online due to 7 day window spcified by $\ui{t}{e}$. The sudden shift in environmental conditions, due to the transfer of the system to outside changed the dynamics of the system's temperature. The shift was subtle and the absolute temperature did not trigger alarm. On the other hand, However, new behavior was adopted by the AID framework within five days, reducing potential false alerts afterward, by observably shifting the conditional mean to lower temperatures. Perhaps more interesting are the alerted changepoint adaptation events.

Calibration of the BESS, usually observed as deviations of setpoint from real power demand and multiple peaks in temperature were captured as well.

The system identified 6 deviations in sampling, denoted by the red bars in Fig. \ref{fig:bess}. 4 anomalies with shorter duration represented packet loss. The prolonged anomaly was notified during the transfer of the battery pack. The longest dropout observed happened across 20\textsuperscript{th} March up to 21\textsuperscript{st}. Unexpectedly, the change point detection module triggered an alarm at the end of the loss, resulting in adaptation and a sharp shift in drawn limits for Power Setpoint Deviation. Red dots represent anomalies at the signal level given by equation \eqref{eq:anomaly_signal}. The dynamic signal limits are surpassed in one or multiple signals during the system's anomalies. The root cause isolation allows the pairing of anomalies with specific features. Conditional probability, against which the anomalies are evaluated allows consideration of signal relationships within individual limits.

\subsection{Kokam Battery Temperature Module}\label{AA:Kokam}
A second case study is concerned with monitoring temperature profiles of individual modules of battery pack deployed at end user. During the operation, a hardware fault of the cooling fan happened. Our industrial partner was interested in finding out, whether such an event could be captured by an anomaly detection system. The data for 12 modules, each coming with 6 channels of measurement were retrieved in 30-second sampling and processed in a streamed manner. We found it informative to compute the deviation of the observed value from the average of all the above-mentioned measurements.

Our anomaly detection system was, once again, initialized with an expiration period of 7 days. The grace period was shortened to 1 day. The threshold value was shifted to a 4 sigma value of 99.977\% to minimize the number of alarms.

In Figure \ref{fig:kokam} we observe 5 days of deviations between the observed temperature measured by channels of module 9 and the average temperature of all modules. After the grace period, we observe multiple system alarms raised by various channels. Until the noon of 22\textsuperscript{nd} August, they seem to be spread out randomly between individual channels. During the late evening of 22\textsuperscript{nd}, anomalies were reported by both channels 4 and 5 for a prolonged period, followed by an anomalous rise in temperature measured by channel 6 early in the morning on 23\textsuperscript{rd} August. The fan fault was observed approximately at 5 pm on 23\textsuperscript{rd} August. Our anomaly detection system instantly raised an alarm, notifying us of anomalous behavior reported by channels 1 - 3. The prolonged duration of the alarm triggered the changepoint alarm approximately 2 hours later. This resulted in a slightly faster adaptation of the system to the new operation under increased temperature. Surprisingly, the temperature decreased during the next day, notifying us of the fan being in operation for a brief period, to fail again 30 minutes later after the battery modules were cooled down to the previous setpoint. The anomaly detection system was triggered once again, although adaptation loosened the region of normal operation to allow itself to adapt. No significant anomalies in sampling were observed during the period.

\begin{figure}[htbp]
 \centerline{\includegraphics{figures/Kokam_thresh.pdf}}
 \caption{Time Series of battery module 9 measurements (blue line) of process variables. The y-axis renders the normalized deviations of temperature from average of all 12 modules. Signal anomalies are marked as red dots. The light red area represents out-of-limits values for individual signals. True fan faults are marked by purple bars. Green bars represent the times, at which changepoint was detected. All the signal anomalies are depicted as orange bars below the graph.}
 \label{fig:kokam}
\end{figure}


\section{Conclusion}
\label{Conclusion}
In this paper, we examine the capacity of adaptive conditional probability distribution to model the normal operation of dynamic systems employing streaming IoT devices and isolate the root cause. The dynamics of the systems are elaborated in the model using Welford's online algorithm with the capacity to update and revert sufficient parameters of underlying multivariate Gaussian distribution in time making it possible to elaborate non-stationarity in the process variables. Moreover, the self-supervision allows protection of the distribution from the effect of outliers and increased speed of adaptation in cases of changes in operation.

We assume the Gaussian distribution of measurements over a bounded time frame related to the system dynamics. We consider such an assumption reasonable, with support of multiple trials where the Kolmogorov-Smirnov test did not reject this hypothesis. The statistical model provides the capacity for the interpretation of the anomalies as extremely deviating observations from the mean vector. Another assumption held in this study is that any anomaly, spatial or temporal, can be transformed in such a way that makes it an outlier given that we expose such effects as features as shown in case studies.

Our approach establishes the system's operation state at the global anomaly level by considering interactions between input measurements and engineered features and computing distance from conditional probability. At the second level, dynamic process limits based on PPF at threshold probability, given multivariate distribution parameters, help isolate the root cause of anomalies. This level serves the diagnostic purpose of the model operation. The individual signals contribute to the global anomaly prediction, while the proposed dynamic limits offer less conservative restrictions on individual process operation. In parallel, the detector allows discrimination of signal losses due to packet drops and sensor malfunctioning.

The ability to detect and identify anomalies in the system, isolate the root cause of anomaly to specific signal or feature, and identify signal losses is shown in two case studies on real data. Unlike many anomaly detection approaches, the proposed AID method does not require historical data or ground truth information about anomalies, relieving general limitations. Moreover, it combines adaptability and interpretability, which is an area yet to be explored.

The benchmark performed on industrial data showed the ability to provide comparable results to other self-learning adaptable anomaly detection methods. This is an important property for our model which allows, in addition, the root cause isolation.
The first case study, performed on real operation data of BESS, examined the battery energy storage system and demonstrated the ability to capture system anomalies and provide less conservative limits to signals. The physical model aided decisions about the normality of the measured temperature of BESS.

The second case study exposed the ability to detect anomalies in the temperature profiles of battery modules within the battery pack, considering measurements made by multiple sensors distributed around the module and the average temperature of all the modules within the pack. Hardware fault observed on this deployed device was captured by our model, giving another proof of its importance in energy storage systems monitoring, where tight temperature control plays a significant role in the safety and profitability of the system.

Future works on the method will include improvement to the change point detection mechanism, decrease in the latency on high dimensional data, and false positive rate reduction, from which general plug-and-play models suffer.


\section*{Additional information}
Our framework is openly accessible on GitHub at the following URL: \url{https://github.com/MarekWadinger/online_outlier_detection}.

\section*{CRediT authorship contribution statement}
\textbf{Marek Wadinger:} Conceptualization; Data curation; Formal analysis; Investigation; Methodology; Resources; Software; Validation; Visualization; Writing - original draft; and Writing - review \& editing. \textbf{Michal Kvasnica:} Conceptualization; Funding acquisition; Project administration; Resources; Supervision; Validation.

\section*{Declaration of Competing Interest}
The authors declare that they have no known competing financial interests or personal relationships that could have appeared to influence the work reported in this paper.

\section*{Acknowledgements}
This work was supported by the Horizon Europe [101079342]; the Slovak Research and Development Agency [APVV-20-0261]; and the Scientific Grant Agency of the Slovak Republic [1/0490/23].

% HORIZON EUROPE
% Agentúra na Podporu Výskumu a Vývoja
% Vedecká grantová agentúra MŠVVaŠ SR a SAV

%% The Appendices part is started with the command \appendix;
%% appendix sections are then done as normal sections
%% \appendix

%% \section{}
%% \label{}

%% For citations use:
%%       \citet{<label>} ==> Jones et al. [21]
%%       \citep{<label>} ==> [21]
%%


\bibliographystyle{elsarticle-harv}
\bibliography{main}

\end{document}

\endinput
