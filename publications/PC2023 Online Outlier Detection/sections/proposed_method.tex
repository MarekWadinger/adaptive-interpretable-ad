We suggest a novel approach to provide dynamic process limits using an online outlier detection algorithm capable of handling concept drifts in real-time. Our main contribution is based on using an inverse cumulative distribution function (ICDF) to supply a dynamic real-valued threshold for anomaly detection, i.e., to find the values of the signal which corresponds to the alert-triggering process limits. Therefore, in the context of machine learning, we are tackling an inverse problem, i.e., calculating the input that produced the observation. To utilize an adaptive ICDF-based threshold system, the univariate Gaussian distribution has to be fitted to the data in online training and ICDF evaluated on the fly. It is important to note that the analysis is based on the assumption that the data collected over moving windows follow a Gaussian normal distribution, rather than assuming that the data over the entire observed period follows this distribution. Thus, the influence of trends in the data can be mitigated by selecting the appropriate window size. 
This method is divided into four parts and described in the following lines. For a simplified representation of the method see Algorithm \ref{alg:detector}.

\subsection{Model Initialization}\label{init}
The initial conditions of the model parameters are \(\mu_0 = x_0\) for mean and \(s^2_0 = 1\) for variance. The score threshold $q$ is constant and set to $3\sigma$. Moreover, there are two user-defined parameters: the expiration period $t_e$, and the time constant of the system $t_c$. The expiration period, which defines the period over which the time-rolling computations are performed, can be altered to change the proportion of expected anomalies and allows relaxation (longer expiration period) or tightening (shorter expiration period) of the thresholds. The time constant of the system determines the speed of change point adaptation as it influences the selection of anomalous points that will be used to update the model for a window of values \(Y=\{y_{i-t_c},...,y_{i}\}\) if the following condition holds true
\begin{equation}
{\frac{\sum_{y\in Y}y}{n(Y)}} > q\text{,}\label{eq:condition}
\end{equation}
where \(n(Y)\) represents dimensionality of \(Y\).

The existence of two tunable and easy-to-interpret hyper-parameters makes it very easy to adapt the solution to any univariate anomaly detection problem.

\subsection{Online training}\label{train}
Training of the model takes place in an online fashion, i.e., the model learns one sample at a time at the moment of its arrival. Learning updates the mean and variance of the underlying Gaussian distribution. The computation of moving mean \eqref{eq:runmean} and variance \eqref{eq:runvar} is handled by Welford's method. Each sample after the expiration period is forgotten and its effect reverted in the second pass. First, the new mean is computed using \eqref{eq:revmean} which accesses the first value in the bounded buffer. The value is dropped in the same pass. Second, the new sample variance is reverted based on \eqref{eq:revvar} using the new mean and current mean that is overwritten afterward. For details see Subsection \ref{AA:InvWelford}.

\subsection{Online prediction}\label{predict}
In the prediction phase, \(z\)-score \eqref{eq:zscore} is computed and passed through $E_A$ \eqref{eq:erf} in order to evaluate $F_{X}(x_i)$ from \eqref{eq:cdf}. The algorithm marks the incoming data points if their corresponding anomaly score from \eqref{eq:score} is out of the range defined by threshold \(q\). In other words, it marks signal value \(x_i\) that is higher or equal to the threshold, which bounds the $3\sigma$ region.
%, i.e., the probability, that the next value \(x\) from measurement real-valued random variable \(X\) will be inside the bounded region \(x\).

\subsection{Dynamic Process Limits}\label{constrait}
Normal process operation is constrained online using ICDF. The constant value of \(q\) and parameters of the fitted distribution are both passed through Algorithm \ref{alg:ppf} to obtain value, which corresponds to the value of \(x\) that would trigger an upper bound outlier alarm at the given time instance. To obtain a lower bound of operation conditions the same procedure is applied to the distribution fitted on negative values of input.

\begin{algorithm}[H]
\caption{{Online Anomaly Detection Workflow}} \label{alg:detector}
 \begin{algorithmic}[1]
  \renewcommand{\algorithmicrequire}{\textbf{Input:}}
  \renewcommand{\algorithmicensure}{\textbf{Output:}}
  \REQUIRE expiration period $t_e$, time constant $t_c$
  % sample mean $\bar x_0$, sample variance $s^2_0$, 
  \ENSURE  score $y_i$, threshold $x_{i,q}$
 \\ \textit{Initialisation} : 
  \STATE $i \leftarrow 1;~ n \leftarrow 1;~ q \leftarrow 0.9973;~ \bar x  \leftarrow x_0;~  s^2 \leftarrow 1$;
  \STATE compute $F_X(x_0)$ using \eqref{eq:zscore};
 \\ \textit{LOOP Process}
  \LOOP
    \STATE {$x_i \leftarrow$ RECEIVE()};
    \STATE $y_i \leftarrow$ PREDICT($x_i$) using \eqref{eq:score};
    \STATE $x_{i,q} \leftarrow$ GET($q, \bar x, s^2$) using Algorithm \ref{alg:ppf};
    \IF {\eqref{eq:score_norm} \OR \eqref{eq:condition}}
     \STATE {$\bar x$, $s^2 \leftarrow$ UPDATE($x_i, \bar x, s^2, n$) using \eqref{eq:runmean}, \eqref{eq:runvar}};
     \STATE $n \leftarrow n + 1$;
     \FOR {$x_{i-t_e}$}
      \STATE {$\bar x$, $s^2 \leftarrow$ REVERT($x_{i-t_e}, \bar x, s^2, n$) using \eqref{eq:revmean}, \eqref{eq:revvar}};
      \STATE $n \leftarrow n - 1$;
     \ENDFOR
    \ENDIF
    \STATE $i \leftarrow i + 1$;
  \ENDLOOP
 \end{algorithmic}
\end{algorithm}