In this paper, we examine the capacity of joint probability distribution to model the normal operation of dynamic systems employing streaming IoT devices. The dynamics of the systems are elaborated in the model using Welford’s online algorithm with the capacity to update and revert sufficient parameters of multivariate Gaussian distribution in time making it possible to elaborate non-stationarity in the process variables.

We assume the Gaussian distribution of measurements over a bounded time frame related to the system dynamics. We consider such an assumption reasonable, with support of multiple trials where the Kolmogorov–Smirnov test did not reject this hypothesis. The statistical model provides the capacity for the interpretation of the anomalies as extremely deviating observations from the mean vector. Another assumption held in this study is that any anomaly, spatial or temporal, can be transformed in such a way that makes it an outlier given one or more interacting signals.

Our approach establishes the system's operation state at the global anomaly level by considering interactions between input measurements and engineered features. At the second level, dynamic process limits based on PPF at threshold probability, given multivariate distribution parameters, help isolate the root cause of anomalies. This level serves the diagnostic purpose of the model operation. The individual signals contribute to the global anomaly prediction, while the proposed dynamic limits offer less conservative restrictions on individual process operation. In parallel, the detector allows discrimination of signal losses due to packet drops and sensor malfunctioning.

The ability to detect and identify anomalies in the system, isolate the root cause of anomaly to specific signal or feature, and identify signal losses is shown in two case studies on real data. Unlike many anomaly detection approaches, the proposed RAID method does not require historical data or ground truth information about anomalies, relieving general limitations.

The first case study performed on benchmark industrial data showed the ability to provide comparable results to other self-learning adaptable anomaly detection methods allowing, in addition, the root cause isolation.
The second case study, performed on real operation data of BESS, examined the battery energy storage system and demonstrated the ability to capture system anomalies and provide less conservative limits to signals and extracted features.

Future works on the method will include improvement to the scalability, a decrease in the latency on high dimensional data, and false positive rate reduction, from which general plug-and-play models suffer.