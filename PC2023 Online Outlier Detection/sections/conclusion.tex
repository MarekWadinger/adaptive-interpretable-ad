This paper proposes a novel approach to real-time anomaly detection that provides a physical threshold that bounds normal process operation. Such an approach has wide applicability in all the process automation fields where low latency evaluation and online adaptation are crucial. Moreover, adaptive operation constraints provide less conservative process limits and govern important insight into systems behavior. The plug-and-play feature of the model makes it easily deployable as shown in two case studies. 

The first case study performed on BESS examined the average battery cell temperature and demonstrated the ability to capture anomalies as well as the capacity to restrict the operational area by inversion of the cumulative distribution function. Following our investigation of state-of-the-art online anomaly detection described in Section \ref{Introduction} we conclude, that although the robustness and performance of complex methods may exceed the performance of the proposed method, the ability to invert the prediction to depict real-time operational restrictions and eschew using non-comprehensible parameters makes it superior for a wide range of use cases. However, the performance might be greatly afflicted when the time constraints of the observed system are not known. This restriction is much weaker than the restriction of the need for data scientists skilled in the hyper-parameter tuning of unsupervised models. Moreover, hyper-parameter tuning calls for ground truth information about anomalies, which requires an exhaustive collection and is not possible in real time. 

Future works on the method will follow three practical challenges:
Firstly, the multivariate online anomaly detection based on the developed method will be researched. The multivariate implementation would allow the detection of temporal anomalies and the use of features that render spatio-temporal characteristics of the modeled system. This is the common property of most of the online anomaly detection methods that do not offer real-valued thresholds on operational conditions. The multivariate clusters can reveal regions of normal operation that would be otherwise detected incorrectly.

Secondly, the challenge of varying positive and negative process limits thresholds will be examined. As depicted in Fig \ref{fig:cs2_threshold} the positive and negative outliers, in many cases, result from different mechanisms that caused them. The current approach draws a range of normal operational conditions centered around the moving mean value.

Thirdly, automated system identification using normal operation data would further simplify the usage by removing the requirement for system dynamics knowledge. The usage of normal distribution makes the three-sigma rule constrain the number of anomalies only theoretically. This allows the number of anomalies in a given time window to vary greatly and thus the performance is not very sensitive to the selection of the threshold. On the contrary, the time window impacts the model's performance.